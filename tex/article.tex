\documentclass[a4paper,10pt,twoside]{article}

% Note that if you want to use the \begin{equation} ... \end{equation}
% environment, you will have to include fleqn in the
% \documentclass[...]{...} options! 
% The top of your LaTeX file should then look like this:
% \documentclass[a4paper,10pt,twoside,fleqn]{article}

\usepackage{clin}        % Stylefile for CLIN Journal
\usepackage{harvard}     % Bibliography Stylefile
\usepackage{wrapfig}
\usepackage{footnote}
\usepackage{hyperref}
%\usepackage{savenotes}
%\usepackage{...,cgloss4e,avm,trees,tree-dvips,gb4e,ipa,graphicx}
                         % Whatever other packages you need
% Harvard:
% \cite{Covington}             (Covington 1994)
% \citeasnoun{Covington}       Covington (1994)
% \citeyear{Covington}         (1994)

\pagestyle{empty}

\begin{document}

\title{Using vector space models based on LDA and LSA for automatically grading exam questions}

%Authors and addresses
%Example has 3 authors with 2 different affiliations
%Adapt where necessary
\author{Johanna F. de Vos$^*$ \email{johannadevos@gmail.com}
\AND \addr{$^*$Radboud University, Donders Institute for Brain, Cognition and Behaviour, Nijmegen, The Netherlands}}


\maketitle\thispagestyle{empty} % extra pagestyle command for first page

%To be filled in by the editors
%Please leave commented out
%\jmlrheading{vol}{year}{pages}{Submission date}{Publication date}{authors}
%\copyright

\begin{abstract}
Students' answers to open-ended exam questions can be automatically graded by comparing them to a `perfect' reference answer in terms of semantic similarity. In this study, I compared three methods for constructing vector spaces in which semantic similarity can be measured, with the aim of finding the most effective automatic grading model. The dimensions of these vector spaces were made up of: 1) Topics, generated by LDA models, versus 2) Singular values, computed by LSA models, versus 3) Vocabulary counts, serving as baseline models. Better results were expected for the LDA and LSA models, as they capture students' answers at the semantic level rather than depending on exact vocabulary matches. Indeed, the best model was one of the LDA models, which achieved a Spearman's correlation coefficient of .52 between the grades predicted by the model and the true grades assigned by the lecturer. The LSA models on average outperformed the baseline models as well. In addition, it was found that the type of data used for training the LDA and LSA models impacted the accuracy of the predictions.
\end{abstract}

\section{Introduction}

Methods for automatically grading answers to exam questions have been developed since the 1960s. Initially, these approaches were based on structural features of student answers, such as the total number of words and the average sentence length (e.g., \citeasnoun{page1966}). In recent years, researchers' focus has shifted to content-based approaches. One such research avenue is based on the concept of semantic similarity: quantifying to what extent two documents are alike in terms of the meaning that they express.

For example, one can compare how similar a student's answer to an exam question is to the `perfect' answer to this question \cite{wolfe1998}, which I will call the `reference answer'. This is the approach that I followed in the current study. The basic idea is that that better student answers will resemble the reference answer more closely than not-as-good student answers. Measures of semantic similarity can be used to quantify how closely a student answer resembles the reference answer, where high semantic similarity indicates that the student's answer deserves a high grade.

The semantic similarity between two documents (here: a student answer and the reference answer) can be quantified by calculating the cosine of the angle between the two vectors that represent these documents in a semantic space. How the dimensions in this semantic space come about depends on the underlying model that is used to capture the meaning of a document. In this project, I focused on two such models: Latent Dirichlet Allocation (LDA) and Latent Semantic Analysis (LSA).

LDA models \cite{blei2003} are generative topic models that take the form of a probability distribution over topics in a document collection. In turn, each topic takes the form of a probability distribution over words. LDA models are calculated from a document-term matrix. When using such a model to calculate semantic similarity between two documents, each dimension in the semantic space represents one topic. More specifically, for each document its vector representation consists of percentages that reflect the proportion of the document concerned with each topic. For example, if we have an LDA model with three topics [`Animals', `Science', `Sports'], the vector for a particular document could be [70, 23, 7].

LSA models \cite{landauer1997} are also topic models that are calculated from a document-term matrix, but they are conceptually and mathematically very different from LDA models. LSA reduces the dimensionality of the document-term matrix. Thus, in the semantic space, the dimensions will be those orthogonal components that are the result of the dimensionality reduction. They cannot be named or interpreted intuitively, like the topics in an LDA model. Nevertheless, we can calculate the distance between two document vectors in this new vector space to obtain the semantic similarity between two documents.

LDA and LSA models have previously been used for grading exam questions. Almost all of these studies focus on LSA (e.g., \citeasnoun{alves2015}, \citeasnoun{tobinski2013}, \citeasnoun{zhang2014}). To my knowledge, only one study \cite{kakkonen2008} has employed LDA for grading exam questions, comparing its effectiveness to that of LSA. \citeasnoun{kakkonen2008} expected LDA to outperform LSA, citing earlier information retrieval studies in which this was the case, but found the opposite. They explain this may be due to the small size of their training data (26, 42 and 70 student essays in three experiments).

The current study is motivated by the scarcity of studies on the use of vector space models in automatic grading in general, as well as by the question of whether \citeasnoun{kakkonen2008}'s training set impacted their results. To investigate the latter issue, I will make use of two different datasets for training, the first one being student answers (like in \citeasnoun{kakkonen2008}), and the second one being a chapter of a psychology textbook, which is more diverse in its contents. Thus, the research questions will be the following:

\begin{enumerate}
\item How effective are LDA-based and LSA-based vector space models for automatically grading students' answers to open exam questions?
\item Does the effectiveness of the above models depend on the training data that is used?
\item Are these topic models more effective than a simple vector space model based on vocabulary counts?
\end{enumerate}

\section{Approach}

\subsection{Data}
The core dataset consisted of 402 students' written answers to a first year psychology exam question from Radboud University. The question was: ``Discuss Whorf's language theory. Include the following terms in your answer: Strong and weak variations on the theory." All answers were accompanied by the grade that the lecturer had assigned to that answer. Permission to work with these data was obtained through the ethical commission of the Social Science faculty. The student answers were used for two purposes: to train topic models (on the training set), and to predict grades from (on the test set). The total number of words was 23467.

As described in the introduction, I also used another text source to train the topic models. This was chapter 10 from \citeasnoun{gleitman2011}, the introductory psychology textbook that the students used. Chapter 10 is the chapter in which Whorf's language theory is explained. The chapter consisted of 1217 sentences (acting as documents in the document-term matrix) and a total of 15096 words.

The reference answer was based on a rubric that was provided by the lecturer, in combination with the definition of Whorf's language theory as given in \citeasnoun{gleitman2011}. It consisted of 59 words in 7 sentences.

\subsection{Preprocessing}
Pre-processing of the data (the student answers and the textbook chapter) was done in Python (version 3.6) with the Natural Language Toolkit (NLTK) (version 3.2.4) \cite{bird2009}. This included tokenization, lemmatization and stop word removal. Manual spelling correction was applied to the student answers (see section \ref{sec:spelling}).

\subsection{Exploring three additional variables}
\label{sec:exploring}
In addition to comparing model types (\textsc{baseline} versus LDA versus LSA) and the influence of the training data (\textsc{student answers} versus the \textsc{textbook chapter}), I also explored three other variables that could be expected to influence the accuracy of the model predictions.

\subsubsection{Counting method}
\label{sec:countingmethod}
I compared \textsc{term frequency-inverse document frequency} (TF-IDF) with \textsc{raw} counts, expecting better results for TF-IDF because it penalizes common words that do not distinguish between topics very well. TF-IDF was not implemented in the LDA models, which require integer counts due to being probabilistic models. I also looked at \textsc{binary} counting (0/1), as the student answers were relatively short and it was conceivable that the presence or absence of certain terms already was informative enough.

\subsubsection{Spelling correction}
\label{sec:spelling}
If no spelling correction were to be applied, misspelled words would get their own vocabulary ID (e.g., `critizing' would be different from `criticizing'), leading to an unnecessary increase in the dimensionality of the dataset. The correct entry `criticizing' would have lower counts, and therefore might be assigned a less important role in the topic models than it should have had in reality. Therefore, I expected better outcomes when the data were spelling-corrected.

\begin{wrapfigure}{r}{5.5cm}
	\vspace{-15pt}
	\includegraphics[width=6cm]{"Histogram of grades"}
	\caption{Frequency distribution of the grades that were assigned by the lecturer (train set only).}
	\label{fig:histogram}
	\vspace{-45pt}
\end{wrapfigure} 

\subsubsection{Mapping algorithm}
Semantic similarity between the student answer and the reference answer was measured in terms of cosine similarity, which always lies between 0 and 1. The most straightforward way to transform this value into a grade is to multiply it by 10. I will call this mapping algorithm `\textsc{x10}'. However, Figure \ref{fig:histogram} shows that the grades of 1 and 9 were almost never assigned by the lecturer, but 0 and 10 quite often. Therefore, to try to improve prediction accuracy, the second mapping algorithm first multiplied the cosine similarity by 10, but subsequently mapped any 1s to 0, and any 9s to 10. I will call this mapping algorithm `\textsc{no 1 or 9}'.

%\begin{figure}[h]
%	\centering
%	\includegraphics[width=0.5\linewidth]{"Histogram of grades"}
%	\caption{Frequency distribution of the grades that were assigned by the lecturer (train split only).}
%	\label{fig:histogram}
%\end{figure} 

\subsection{Implementation}

\subsubsection{Baseline}
The baseline models were vector space models whose dimensions corresponded to the words in the vocabulary of the test data and the reference answer. The way in which this vocabulary was counted is explained in section \ref{sec:countingmethod}. Thus, the baseline models were no topic models and were not trained on any data.

\subsubsection{Topic models}
The LDA models were instances of the LdaModel class in Python's gensim library \cite{rehurek2010}, and the LSA models were instances of gensim's LsiModel class. The number of topics (in LDA) or singular values (SVs) (in LSA) needs to be specified before training. To find the optimal value for these hyperparameters, I performed a grid search on the training data.

\subsubsection{Grid search}
\label{sec:gridsearch}
The student answers were split into a 80/20 train/test set. Ten-fold cross-validation was applied to the training set in order to find the optimal number of topics and SVs. The models were evaluated by predicting grades for the answers in the validation subset of the training set (see section \ref{sec:evaluation}). I used this optimal number of topics or SVs to train the final topic models on all of the training data.

The textbook chapter was not split into a training and test set because it was only used for training topic models, and did not contain any grades that could be predicted. Again, various topic models were trained, differing in their number of topics or SVs. These models were evaluated by predicting grades for all student answers in the training set. Later, I used the best-performing models to predict grades on the test set.

\subsubsection{Training the final models}
\label{sec:training}
As explained in section \ref{sec:exploring}, in this study I investigated/explored five different variables. The main variable under investigation was the model type (LDA versus LSA versus \textsc{baseline}). In addition, we were interested in the training data type (\textsc{student answers} versus a \textsc{textbook chapter}), the counting method (\textsc{raw} versus \textsc{binary} versus TF-IDF), the usage of spelling correction (\textsc{yes} versus \textsc{no}) and the mapping algorithm (\textsc{x10} versus \textsc{no 1 or 9}).

I trained models for all combinations of those five variables, with a few exceptions. Since the baseline models were not trained, for those models the training data type could not be manipulated. Furthermore, TF-IDF was not implemented in the LDA models, which require integer counts due to being probabilistic models. The LSA models could not be trained with TF-IDF when using the textbook data; for unknown reasons the Python kernel always shut down without outputting any error message. In the end, 48 different models were trained that represented all the other variable combinations.

\subsection{Evaluation}
\label{sec:evaluation}
The performance of all models was evaluated by correlating the grades predicted for the student answers in the test data with the grades that the lecturer had assigned to these answers. Spearman's correlation coefficient ($\textit{r}_s$) was used because the assigned and predicted grades often were not normally distributed.

\section{Results}

\subsection{Grid search: optimal number of topics / SVs}
The LDA models that were trained on the \textsc{student answers} on average performed best during ten-fold cross-validation when they contained only 2 topics ($\textit{r}_s$ = .39), as compared to models with 4 or 7 topics. I say `on average' because I averaged the outcomes of those models that varied in counting method, spelling correction and mapping algorithm. When training the LDA models on the \textsc{textbook chapter}, the average outcomes were quite different: the best-performing LDA models contained 20 topics ($\textit{r}_s$ = .49), as compared to 2, 4, 7, 12 or 40 topics.

The LSA models on average did best with 100 SVs ($\textit{r}_s$ = .38) when being trained on the \textsc{student answers}, as compared to 20, 50 or 200 SVs. When training them on the \textsc{textbook chapter}, 10 SVs were used in the best average performance ($\textit{r}_s$ = .35), as compared to 5, 20, 50, 100 or 200 SVs. The outcomes reported below are the result of models that were trained with the here-reported optimal values.

\subsection{Models}

\subsubsection{Best-performing models}
The outcomes for all 48 individual models are listed in Table \ref{all-models} in the Appendix at the end of this article. The best-performing LDA model achieved a correlation of $\textit{r}_s$ = .52. It was trained on the \textsc{textbook chapter}, used \textsc{binary} counting, was \textsc{spelling-corrected} and used the \textsc{x10} mapping algorithm. The scatterplot of predicted and assigned grades is shown in Figure \ref{lda}.
Bigger dots represent more data points.

\begin{figure}[h]
\centering
\includegraphics[width=0.5\linewidth]{"Best LDA model"}
\caption{Grades predicted by the best LDA model.}
\label{lda}
\end{figure}

The best-performing LSA model achieved a correlation of $\textit{r}_s$ = .42. It was trained on \textsc{non-spelling-corrected} \textsc{student answers} and used \textsc{binary} counting. The mapping algorithm was not relevant, as no 1s and 9s were predicted to begin with. Figure \ref{lsa} shows the scatterplot for this model.

\begin{figure}[h]
\centering
\includegraphics[width=0.5\linewidth]{"Best LSA model"}
\caption{Grades predicted by the best LSA model.}
\label{lsa}
\end{figure}

Both of these models outperformed the best \textsc{baseline} model, which achieved a correlation of $\textit{r}_s$ = .38, using \textsc{raw} counts and \textsc{spelling-corrected} data. As in the LSA model, the mapping algorithm was irrelevant. The scatterplot is shown in Figure \ref{baseline}.

\begin{figure}[h]
\centering
\includegraphics[width=0.5\linewidth]{"Best baseline model"}
\caption{Grades predicted by the best baseline model.}
\label{baseline}
\end{figure}

These three highest correlations all were significant at the .05 level (all \textit{p} $<$ .001).

\subsubsection{Outcomes per variable}
In the previous section, I focused on individual models. In this section, I averaged over the outcomes of the individual models as reported in the Appendix, in order to investigate the average impact of model type, training data type, counting method, spelling correction, and the mapping algorithm. 

Table \ref{table1} shows the average correlation that was obtained on the test data between predicted and lecturer-assigned grades, for each model type. The first column contains the four other variables under investigation. For each variable, the results in columns 3-5 are averaged over the levels (column 2) of the other three variables. For example, the results for the two types of training data are averaged over the different levels of counting method, spelling correction and mapping algorithm.\footnote{Because the LSA models could not be trained with TF-IDF when using the textbook data (see section \ref{sec:training}), that cell specifically contains the correlation when using \textsc{student answers} as training data. In all other cells in the LSA column as well, this specific combination of variable levels is missing from the average.} It is possible that there could be (an) interaction(s) between the variables, but that was outside the scope of this study.

As can be seen from Table \ref{table1}, on average, the LDA models performed best, followed by the LSA models. The baseline models on averaged performed worst. Which training data type yielded the best models, seems to depend on model type: LSA models on average yielded higher correlations when being trained on \textsc{student answers}, whereas LDA models did better when trained on the \textsc{textbook chapter}.

\begin{table}[h]
	\caption{Correlations for all model types and other variables.}
	\label{table1}
	\centering
	\begin{tabular}{lllll}
		\hline  &  & \textbf{Baseline} & \textbf{LDA} & \textbf{LSA} \\ 
		\hline  Training data & \textsc{Student answers} & N/A & .36 & .39 \\ 
		& \textsc{Textbook chapter} & N/A & .42 & .32 \\ 
		\hline  Counting method & \textsc{Raw} & .38 & .38 & .34 \\ 
		& \textsc{Binary} & .34 & .40 & .35 \\ 
		& TF-IDF & .30 & N/A & .41 \\ 
		\hline  Spelling correction & \textsc{Yes} & .34 & .37 & .36 \\ 
		& \textsc{No} & .34 & .41 & .36 \\ 
		\hline  Mapping algorithm & \textsc{x10} & .35 & .36 & .36 \\ 
		& \textsc{No 1 or 9} & .33 & .42 & .36 \\ 
		\hline  \textbf{Overall} &  & \textbf{.34} & \textbf{.39} & \textbf{.36} \\ 
		\hline 
	\end{tabular} 
\end{table}

%\newpage
\section{Discussion}
In this study I compared three types of vector space models for automatically predicting exam grades. The vector space models that used topics (LDA) or singular values (LSA) as their dimensions both outperformed the \textsc{baseline} vector space models that used vocabulary counts as their dimensions. The likely explanation for this is that LDA and LSA models compare documents at the semantic level, rather than looking for exact vocabulary matches \cite{kakkonen2008}. Notably, even in all of the baseline models the correlation between predicted and lecturer-assigned grades was statistically significant (see Table \ref{all-models}).

It should be noted that in the results section as well as in the remainder of this discussion, I have made comparisons between models that differed in their dimensionality. For example, the LDA models contained 2 topics when being trained on the \textsc{student answers}, but 20 when being trained on the \textsc{textbook chapter}. These values were chosen because a grid search had shown that they resulted in the most accurate models. However, technically speaking, the outcomes of models that differ in their dimensionality cannot directly be compared, because vectors with lower dimensionalities by default are expected to be more similar. Still, this does not mean that models with lower dimensionalities will \textit{automatically} make the best predictions, which can also be seen from the finding that the best-scoring model was an LDA model with 20 topics. Because I wanted to find the best possible vector space for doing automatic grading, I opted to work with models of different dimensionalities. The readers should keep this in mind when interpreting the results.

\subsection{LDA versus LSA, as a function of training data type}
Looking back on the literature, \citeasnoun{kakkonen2008} asked whether their LSA models perhaps outperformed their LDA models because they were trained on a small set of student answers only. My results support this explanation: when training on the \textsc{student answers}, LSA outperformed LDA, but it was the other way around when using a more diverse document collection (i.e., sentences in a psychology \textsc{textbook chapter}). This raises the question as to why the two types of training data affected the topic models differently (while the size of the training sets was similar).

One explanation can be sought in the optimal number of topics for the LDA models that resulted from the grid search: two when training on \textsc{student answers}, and 20 when training on the \textsc{textbook chapter}. Two is the lowest possible number of topics that an LDA model can have. It is not so unexpected that the grid search yielded this as the best value when training on \textsc{student answers}, as in essence the student answers all revolved around the same topic (i.e., Whorf's language theory). As a result, however, the resulting vector space models only had two dimensions, which does not allow very fine distinctions between answers of different qualities. When visually inspecting the scatter plots of the outcomes of the LDA models with 2 topics trained on \textsc{student answers}, it can be seen that these models predict a 10 for many students, which results in relatively low correlations. As an example, the outcomes of one such model are printed as Figure \ref{lda-predicts-tens} below. For this particular model, $\textit{r}_s$ = .28.

\begin{figure}[h]
	\centering
	\includegraphics[width=0.5\linewidth]{"LDA predicts tens"}
	\caption{Grades predicted by an LDA model that was trained on the \textsc{student answers}, used \textsc{raw} counting and used the \textsc{x10} mapping algorithm.}
	\label{lda-predicts-tens}
\end{figure}

Thus, training on a more diverse corpus such as a textbook seems more suitable when working with LDA models. Indeed, Table 1 showed better results for LDA models that were trained on the \textsc{textbook chapter}, as compared to LDA models that were trained on \textsc{student answers}. The textbook-trained LDA models also outperformed the textbook-trained LSA models by a large margin. This is the outcome that \citeasnoun{kakkonen2008} had predicted in their own study. They list several arguments as to why LDA models are expected to function better, including the difficulty to select the right number of dimensions for LSA (although this was to some extent covered by the grid search), the fact that no probability distribution is defined in LSA, and that the reduced matrix in LSA can contain negative values. These explanations likely also apply to the current study. Incidentally, it is nice to see the LDA model achieving the best results, because this is the model whose outcomes are easiest to interpret by students and lecturers.

\subsection{Spelling correction, mapping algorithm, and counting method}
Spelling correction and the mapping algorithm did not seem to have much of an effect in the baseline and LSA models. A potential explanation is that the students did not make many spelling errors: Dutch students generally are quite proficient in English. Regarding the the mapping algorithm, this variable can only exert an effect on the outcomes when the models predict the grades 1 and 9 to begin with. This was often not the case (or, alternatively, very few 1s and 9s were predicted), which likely explains the lack of effect of this variable. Spelling correction and mapping algorithm did seem to have some effect in the LDA models (see Table \ref{table1}). However, among the three model types the LDA models are the least reliable for evaluating such variables, because they are probabilistic. In fact, their results were seen to fluctuate quite a bit when being trained multiple times with the same hyperparameter values.

Regarding the counting method, the \textsc{baseline} model worked best with raw counts; for LDA not much of a difference was seen, and for LSA it seems best to use TF-IDF counting, although the latter cannot be said with much certainty either due to the issue described in section \ref{sec:training}. Therefore, the investigation of the influence of counting method should be considered only exploratory. To gain more insight in this variable, the computational issue should first be fixed. In addition, to obtain more reliable outcomes for the LDA models, it would be good if they were trained many times and their outcomes averaged. 

\subsection{Are the best models good enough?}
I will now discuss the best individual models, whose outcomes were visually represented in Figures \ref{lda}, \ref{lsa} and \ref{baseline}. Among other things, these figures showed that none of the best models ever predicted the grade of 10, even though this was the most common grade in the training set (see Figure 1). Thus, while the lecturer considered many of the students' answers to be excellent, these answers apparently still were quite different from the reference answer in terms of the vector space models that I used for calculating semantic similarity. This shows that \citeasnoun{wolfe1998}'s reference answer approach may not be ideal for grading student answers by means of vector space models.

An alternative approach is described in \citeasnoun{foltz1998}, which may provide a better solution for future research. Rather than comparing a student's answer to one specific reference answer, the to-be-graded answer is compared to a set of answers already graded by the lecturer. Then, you take the (for example) ten graded answers with the highest similarity score to the to-be-graded answer. The predicted grade will be the weighted average of the grades of the ten most similar answers (weighted by their similarity score). Another alternative is to use a variety of different reference answers. These could even be obtained automatically by applying automatic paraphrasing to the original reference answer.

The approaches discussed so far, including the one by \citeasnoun{foltz1998}, are all limited in the fact that creative answers are punished, such as when a student uses a unique example. This makes the student's answer more dissimilar from the reference answer (or from other students' answers), and would result in a lower grade. One solution could be to create a list with terms that are relevant to the question (e.g., [`Whorf', `linguistic', `relativity']) and first reduce the student's answer to only those sentences that contain one of these terms. Sentences such as `The Inuit have many words for snow' would no longer be part of the semantic similarity calculation, and would no longer wrongfully lower the predicted grade.

Even given the limitations described above, a model does not have to be perfect in order to be of potential use. For example, upon inspecting the individual models' results, I could see that whenever the model had assigned a zero to a student answer, the lecturer had also assigned a zero. Thus, the current models \textit{are} good enough to automatically detect which answers are worth zero points. A lecturer could use this information to focus his/her attention only on those answers that have not been predicted a zero grade by the model.

Before automatic grading models could actually be implemented in higher education for grading \textit{all} students' answers, we would need to know what is the correlation between two human graders, and equal or go beyond that correlation with our best model. Furthermore, the model would need to be able to assign the maximum grade of 10 to excellent answers that are different from the reference answer. With the suggestions proposed in this discussion, we should continue to try to find a reliable method for automatizing the grading process. This could be of great benefit to students and lecturers alike, as students could be provided with feedback much more quickly, and lecturers would see their workload reduced.

\section*{Source code and data}
The source code and data to replicate the analysis can be found on GitHub: \url{https://github.com/johannadevos/AutomaticGrading}
%\nocite{Sag} % items in your bibliography file that are not cited in your text

% add other references to the file bibliography.bib in this directory
\bibliographystyle{clin} 
\bibliography{bibliography} 

\newpage
\section*{Appendix}
\begin{table}[h]
	\centering
	\footnotesize
	\caption{All models. The best correlation(s) per model type is/are highlighted in boldface.}
	\label{all-models}
	\begin{tabular}{lllllll}
		\hline model type & training data type & counting method & spelling correction & mapping algorithm & $\textit{r}_s$   & \textit{p} \\
		\hline baseline   & none               & raw             & yes                 & x10               & \textbf{.38} & .0005       \\
		baseline   & none               & raw             & no                  & x10               & .37 & .0006       \\
		baseline   & none               & raw             & yes                 & no 1 or 9         & \textbf{.38} & .0005       \\
		baseline   & none               & raw             & no                  & no 1 or 9         & .37 & .0006       \\
		baseline   & none               & binary          & yes                 & x10               & .33 & .0025       \\
		baseline   & none               & binary          & no                  & x10               & .36 & .0010       \\
		baseline   & none               & binary          & yes                 & no 1 or 9         & .33 & .0029       \\
		baseline   & none               & binary          & no                  & no 1 or 9         & .35 & .0013       \\
		baseline   & none               & TF-IDF          & yes                 & x10               & .34 & .0022       \\
		baseline   & none               & TF-IDF          & no                  & x10               & .33 & .0029       \\
		baseline   & none               & TF-IDF          & yes                 & no 1 or 9         & .27 & .0167       \\
		baseline   & none               & TF-IDF          & no                  & no 1 or 9         & .25 & .0244       \\
		LDA        & student answers    & raw             & yes                 & x10               & .28 & .0100       \\
		LDA        & student answers    & raw             & no                  & x10               & .21 & .0547       \\
		LDA        & student answers    & raw             & yes                 & no 1 or 9         & .43 & $<$ .0001       \\
		LDA        & student answers    & raw             & no                  & no 1 or 9         & .47 & $<$ .0001       \\
		LDA        & student answers    & binary          & yes                 & x10               & .26 & .0200       \\
		LDA        & student answers    & binary          & no                  & x10               & .29 & .0085       \\
		LDA        & student answers    & binary          & yes                 & no 1 or 9         & .45 & $<$ .0001       \\
		LDA        & student answers    & binary          & no                  & no 1 or 9         & .45 & $<$ .0001       \\
		LDA        & textbook chapter   & raw             & yes                 & x10               & .44 & $<$ .0001       \\
		LDA        & textbook chapter   & raw             & no                  & x10               & .45 & $<$ .0001       \\
		LDA        & textbook chapter   & raw             & yes                 & no 1 or 9         & .27 & .0134       \\
		LDA        & textbook chapter   & raw             & no                  & no 1 or 9         & .48 & $<$ .0001       \\
		LDA        & textbook chapter   & binary          & yes                 & x10               & \textbf{.52} & $<$ .0001       \\
		LDA        & textbook chapter   & binary          & no                  & x10               & .39 & .0004       \\
		LDA        & textbook chapter   & binary          & yes                 & no 1 or 9         & .30 & .0066       \\
		LDA        & textbook chapter   & binary          & no                  & no 1 or 9         & .51 & $<$ .0001       \\
		LSA        & student answers    & raw             & yes                 & x10               & .36 & .0011       \\
		LSA        & student answers    & raw             & no                  & x10               & .34 & .0021       \\
		LSA        & student answers    & raw             & yes                 & no 1 or 9         & .36 & .0011       \\
		LSA        & student answers    & raw             & no                  & no 1 or 9         & .34 & .0021       \\
		LSA        & student answers    & binary          & yes                 & x10               & .40 & .0003       \\
		LSA        & student answers    & binary          & no                  & x10               & \textbf{.42} & $<$ .0001       \\
		LSA        & student answers    & binary          & yes                 & no 1 or 9         & .40 & .0003       \\
		LSA        & student answers    & binary          & no                  & no 1 or 9         & \textbf{.42} & $<$ .0001       \\
		LSA        & student answers    & TF-IDF          & yes                 & x10               & .41 & .0001       \\
		LSA        & student answers    & TF-IDF          & no                  & x10               & .40 & .0002       \\
		LSA        & student answers    & TF-IDF          & yes                 & no 1 or 9         & .41 & .0001       \\
		LSA        & student answers    & TF-IDF          & no                  & no 1 or 9         & .40 & .0002       \\
		LSA        & textbook chapter   & raw             & yes                 & x10               & .32 & .0034       \\
		LSA        & textbook chapter   & raw             & no                  & x10               & .33 & .0029       \\
		LSA        & textbook chapter   & raw             & yes                 & no 1 or 9         & .33 & .0027       \\
		LSA        & textbook chapter   & raw             & no                  & no 1 or 9         & .32 & .0033       \\
		LSA        & textbook chapter   & binary          & yes                 & x10               & .29 & .0083       \\
		LSA        & textbook chapter   & binary          & no                  & x10               & .32 & .0040       \\
		LSA        & textbook chapter   & binary          & yes                 & no 1 or 9         & .29 & .0081       \\
		LSA        & textbook chapter   & binary          & no                  & no 1 or 9         & .32 & .0039 \\
		\hline     
	\end{tabular}
\end{table}





\end{document}
